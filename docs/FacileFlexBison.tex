
% Default to the notebook output style

    


% Inherit from the specified cell style.




    
\documentclass[11pt]{article}

    
    
    \usepackage[T1]{fontenc}
    % Nicer default font (+ math font) than Computer Modern for most use cases
    \usepackage{mathpazo}

    % Basic figure setup, for now with no caption control since it's done
    % automatically by Pandoc (which extracts ![](path) syntax from Markdown).
    \usepackage{graphicx}
    % We will generate all images so they have a width \maxwidth. This means
    % that they will get their normal width if they fit onto the page, but
    % are scaled down if they would overflow the margins.
    \makeatletter
    \def\maxwidth{\ifdim\Gin@nat@width>\linewidth\linewidth
    \else\Gin@nat@width\fi}
    \makeatother
    \let\Oldincludegraphics\includegraphics
    % Set max figure width to be 80% of text width, for now hardcoded.
    \renewcommand{\includegraphics}[1]{\Oldincludegraphics[width=.8\maxwidth]{#1}}
    % Ensure that by default, figures have no caption (until we provide a
    % proper Figure object with a Caption API and a way to capture that
    % in the conversion process - todo).
    \usepackage{caption}
    \DeclareCaptionLabelFormat{nolabel}{}
    \captionsetup{labelformat=nolabel}

    \usepackage{adjustbox} % Used to constrain images to a maximum size 
    \usepackage{xcolor} % Allow colors to be defined
    \usepackage{enumerate} % Needed for markdown enumerations to work
    \usepackage{geometry} % Used to adjust the document margins
    \usepackage{amsmath} % Equations
    \usepackage{amssymb} % Equations
    \usepackage{textcomp} % defines textquotesingle
    % Hack from http://tex.stackexchange.com/a/47451/13684:
    \AtBeginDocument{%
        \def\PYZsq{\textquotesingle}% Upright quotes in Pygmentized code
    }
    \usepackage{upquote} % Upright quotes for verbatim code
    \usepackage{eurosym} % defines \euro
    \usepackage[mathletters]{ucs} % Extended unicode (utf-8) support
    \usepackage[utf8x]{inputenc} % Allow utf-8 characters in the tex document
    \usepackage{fancyvrb} % verbatim replacement that allows latex
    \usepackage{grffile} % extends the file name processing of package graphics 
                         % to support a larger range 
    % The hyperref package gives us a pdf with properly built
    % internal navigation ('pdf bookmarks' for the table of contents,
    % internal cross-reference links, web links for URLs, etc.)
    \usepackage{hyperref}
    \usepackage{longtable} % longtable support required by pandoc >1.10
    \usepackage{booktabs}  % table support for pandoc > 1.12.2
    \usepackage[inline]{enumitem} % IRkernel/repr support (it uses the enumerate* environment)
    \usepackage[normalem]{ulem} % ulem is needed to support strikethroughs (\sout)
                                % normalem makes italics be italics, not underlines
    \usepackage{mathrsfs}
    

    
    
    % Colors for the hyperref package
    \definecolor{urlcolor}{rgb}{0,.145,.698}
    \definecolor{linkcolor}{rgb}{.71,0.21,0.01}
    \definecolor{citecolor}{rgb}{.12,.54,.11}

    % ANSI colors
    \definecolor{ansi-black}{HTML}{3E424D}
    \definecolor{ansi-black-intense}{HTML}{282C36}
    \definecolor{ansi-red}{HTML}{E75C58}
    \definecolor{ansi-red-intense}{HTML}{B22B31}
    \definecolor{ansi-green}{HTML}{00A250}
    \definecolor{ansi-green-intense}{HTML}{007427}
    \definecolor{ansi-yellow}{HTML}{DDB62B}
    \definecolor{ansi-yellow-intense}{HTML}{B27D12}
    \definecolor{ansi-blue}{HTML}{208FFB}
    \definecolor{ansi-blue-intense}{HTML}{0065CA}
    \definecolor{ansi-magenta}{HTML}{D160C4}
    \definecolor{ansi-magenta-intense}{HTML}{A03196}
    \definecolor{ansi-cyan}{HTML}{60C6C8}
    \definecolor{ansi-cyan-intense}{HTML}{258F8F}
    \definecolor{ansi-white}{HTML}{C5C1B4}
    \definecolor{ansi-white-intense}{HTML}{A1A6B2}
    \definecolor{ansi-default-inverse-fg}{HTML}{FFFFFF}
    \definecolor{ansi-default-inverse-bg}{HTML}{000000}

    % commands and environments needed by pandoc snippets
    % extracted from the output of `pandoc -s`
    \providecommand{\tightlist}{%
      \setlength{\itemsep}{0pt}\setlength{\parskip}{0pt}}
    \DefineVerbatimEnvironment{Highlighting}{Verbatim}{commandchars=\\\{\}}
    % Add ',fontsize=\small' for more characters per line
    \newenvironment{Shaded}{}{}
    \newcommand{\KeywordTok}[1]{\textcolor[rgb]{0.00,0.44,0.13}{\textbf{{#1}}}}
    \newcommand{\DataTypeTok}[1]{\textcolor[rgb]{0.56,0.13,0.00}{{#1}}}
    \newcommand{\DecValTok}[1]{\textcolor[rgb]{0.25,0.63,0.44}{{#1}}}
    \newcommand{\BaseNTok}[1]{\textcolor[rgb]{0.25,0.63,0.44}{{#1}}}
    \newcommand{\FloatTok}[1]{\textcolor[rgb]{0.25,0.63,0.44}{{#1}}}
    \newcommand{\CharTok}[1]{\textcolor[rgb]{0.25,0.44,0.63}{{#1}}}
    \newcommand{\StringTok}[1]{\textcolor[rgb]{0.25,0.44,0.63}{{#1}}}
    \newcommand{\CommentTok}[1]{\textcolor[rgb]{0.38,0.63,0.69}{\textit{{#1}}}}
    \newcommand{\OtherTok}[1]{\textcolor[rgb]{0.00,0.44,0.13}{{#1}}}
    \newcommand{\AlertTok}[1]{\textcolor[rgb]{1.00,0.00,0.00}{\textbf{{#1}}}}
    \newcommand{\FunctionTok}[1]{\textcolor[rgb]{0.02,0.16,0.49}{{#1}}}
    \newcommand{\RegionMarkerTok}[1]{{#1}}
    \newcommand{\ErrorTok}[1]{\textcolor[rgb]{1.00,0.00,0.00}{\textbf{{#1}}}}
    \newcommand{\NormalTok}[1]{{#1}}
    
    % Additional commands for more recent versions of Pandoc
    \newcommand{\ConstantTok}[1]{\textcolor[rgb]{0.53,0.00,0.00}{{#1}}}
    \newcommand{\SpecialCharTok}[1]{\textcolor[rgb]{0.25,0.44,0.63}{{#1}}}
    \newcommand{\VerbatimStringTok}[1]{\textcolor[rgb]{0.25,0.44,0.63}{{#1}}}
    \newcommand{\SpecialStringTok}[1]{\textcolor[rgb]{0.73,0.40,0.53}{{#1}}}
    \newcommand{\ImportTok}[1]{{#1}}
    \newcommand{\DocumentationTok}[1]{\textcolor[rgb]{0.73,0.13,0.13}{\textit{{#1}}}}
    \newcommand{\AnnotationTok}[1]{\textcolor[rgb]{0.38,0.63,0.69}{\textbf{\textit{{#1}}}}}
    \newcommand{\CommentVarTok}[1]{\textcolor[rgb]{0.38,0.63,0.69}{\textbf{\textit{{#1}}}}}
    \newcommand{\VariableTok}[1]{\textcolor[rgb]{0.10,0.09,0.49}{{#1}}}
    \newcommand{\ControlFlowTok}[1]{\textcolor[rgb]{0.00,0.44,0.13}{\textbf{{#1}}}}
    \newcommand{\OperatorTok}[1]{\textcolor[rgb]{0.40,0.40,0.40}{{#1}}}
    \newcommand{\BuiltInTok}[1]{{#1}}
    \newcommand{\ExtensionTok}[1]{{#1}}
    \newcommand{\PreprocessorTok}[1]{\textcolor[rgb]{0.74,0.48,0.00}{{#1}}}
    \newcommand{\AttributeTok}[1]{\textcolor[rgb]{0.49,0.56,0.16}{{#1}}}
    \newcommand{\InformationTok}[1]{\textcolor[rgb]{0.38,0.63,0.69}{\textbf{\textit{{#1}}}}}
    \newcommand{\WarningTok}[1]{\textcolor[rgb]{0.38,0.63,0.69}{\textbf{\textit{{#1}}}}}
    
    
    % Define a nice break command that doesn't care if a line doesn't already
    % exist.
    \def\br{\hspace*{\fill} \\* }
    % Math Jax compatibility definitions
    \def\gt{>}
    \def\lt{<}
    \let\Oldtex\TeX
    \let\Oldlatex\LaTeX
    \renewcommand{\TeX}{\textrm{\Oldtex}}
    \renewcommand{\LaTeX}{\textrm{\Oldlatex}}
    % Document parameters
    % Document title
    \title{FacileFlexBison}
    
    
    
    
    

    % Pygments definitions
    
\makeatletter
\def\PY@reset{\let\PY@it=\relax \let\PY@bf=\relax%
    \let\PY@ul=\relax \let\PY@tc=\relax%
    \let\PY@bc=\relax \let\PY@ff=\relax}
\def\PY@tok#1{\csname PY@tok@#1\endcsname}
\def\PY@toks#1+{\ifx\relax#1\empty\else%
    \PY@tok{#1}\expandafter\PY@toks\fi}
\def\PY@do#1{\PY@bc{\PY@tc{\PY@ul{%
    \PY@it{\PY@bf{\PY@ff{#1}}}}}}}
\def\PY#1#2{\PY@reset\PY@toks#1+\relax+\PY@do{#2}}

\expandafter\def\csname PY@tok@w\endcsname{\def\PY@tc##1{\textcolor[rgb]{0.73,0.73,0.73}{##1}}}
\expandafter\def\csname PY@tok@c\endcsname{\let\PY@it=\textit\def\PY@tc##1{\textcolor[rgb]{0.25,0.50,0.50}{##1}}}
\expandafter\def\csname PY@tok@cp\endcsname{\def\PY@tc##1{\textcolor[rgb]{0.74,0.48,0.00}{##1}}}
\expandafter\def\csname PY@tok@k\endcsname{\let\PY@bf=\textbf\def\PY@tc##1{\textcolor[rgb]{0.00,0.50,0.00}{##1}}}
\expandafter\def\csname PY@tok@kp\endcsname{\def\PY@tc##1{\textcolor[rgb]{0.00,0.50,0.00}{##1}}}
\expandafter\def\csname PY@tok@kt\endcsname{\def\PY@tc##1{\textcolor[rgb]{0.69,0.00,0.25}{##1}}}
\expandafter\def\csname PY@tok@o\endcsname{\def\PY@tc##1{\textcolor[rgb]{0.40,0.40,0.40}{##1}}}
\expandafter\def\csname PY@tok@ow\endcsname{\let\PY@bf=\textbf\def\PY@tc##1{\textcolor[rgb]{0.67,0.13,1.00}{##1}}}
\expandafter\def\csname PY@tok@nb\endcsname{\def\PY@tc##1{\textcolor[rgb]{0.00,0.50,0.00}{##1}}}
\expandafter\def\csname PY@tok@nf\endcsname{\def\PY@tc##1{\textcolor[rgb]{0.00,0.00,1.00}{##1}}}
\expandafter\def\csname PY@tok@nc\endcsname{\let\PY@bf=\textbf\def\PY@tc##1{\textcolor[rgb]{0.00,0.00,1.00}{##1}}}
\expandafter\def\csname PY@tok@nn\endcsname{\let\PY@bf=\textbf\def\PY@tc##1{\textcolor[rgb]{0.00,0.00,1.00}{##1}}}
\expandafter\def\csname PY@tok@ne\endcsname{\let\PY@bf=\textbf\def\PY@tc##1{\textcolor[rgb]{0.82,0.25,0.23}{##1}}}
\expandafter\def\csname PY@tok@nv\endcsname{\def\PY@tc##1{\textcolor[rgb]{0.10,0.09,0.49}{##1}}}
\expandafter\def\csname PY@tok@no\endcsname{\def\PY@tc##1{\textcolor[rgb]{0.53,0.00,0.00}{##1}}}
\expandafter\def\csname PY@tok@nl\endcsname{\def\PY@tc##1{\textcolor[rgb]{0.63,0.63,0.00}{##1}}}
\expandafter\def\csname PY@tok@ni\endcsname{\let\PY@bf=\textbf\def\PY@tc##1{\textcolor[rgb]{0.60,0.60,0.60}{##1}}}
\expandafter\def\csname PY@tok@na\endcsname{\def\PY@tc##1{\textcolor[rgb]{0.49,0.56,0.16}{##1}}}
\expandafter\def\csname PY@tok@nt\endcsname{\let\PY@bf=\textbf\def\PY@tc##1{\textcolor[rgb]{0.00,0.50,0.00}{##1}}}
\expandafter\def\csname PY@tok@nd\endcsname{\def\PY@tc##1{\textcolor[rgb]{0.67,0.13,1.00}{##1}}}
\expandafter\def\csname PY@tok@s\endcsname{\def\PY@tc##1{\textcolor[rgb]{0.73,0.13,0.13}{##1}}}
\expandafter\def\csname PY@tok@sd\endcsname{\let\PY@it=\textit\def\PY@tc##1{\textcolor[rgb]{0.73,0.13,0.13}{##1}}}
\expandafter\def\csname PY@tok@si\endcsname{\let\PY@bf=\textbf\def\PY@tc##1{\textcolor[rgb]{0.73,0.40,0.53}{##1}}}
\expandafter\def\csname PY@tok@se\endcsname{\let\PY@bf=\textbf\def\PY@tc##1{\textcolor[rgb]{0.73,0.40,0.13}{##1}}}
\expandafter\def\csname PY@tok@sr\endcsname{\def\PY@tc##1{\textcolor[rgb]{0.73,0.40,0.53}{##1}}}
\expandafter\def\csname PY@tok@ss\endcsname{\def\PY@tc##1{\textcolor[rgb]{0.10,0.09,0.49}{##1}}}
\expandafter\def\csname PY@tok@sx\endcsname{\def\PY@tc##1{\textcolor[rgb]{0.00,0.50,0.00}{##1}}}
\expandafter\def\csname PY@tok@m\endcsname{\def\PY@tc##1{\textcolor[rgb]{0.40,0.40,0.40}{##1}}}
\expandafter\def\csname PY@tok@gh\endcsname{\let\PY@bf=\textbf\def\PY@tc##1{\textcolor[rgb]{0.00,0.00,0.50}{##1}}}
\expandafter\def\csname PY@tok@gu\endcsname{\let\PY@bf=\textbf\def\PY@tc##1{\textcolor[rgb]{0.50,0.00,0.50}{##1}}}
\expandafter\def\csname PY@tok@gd\endcsname{\def\PY@tc##1{\textcolor[rgb]{0.63,0.00,0.00}{##1}}}
\expandafter\def\csname PY@tok@gi\endcsname{\def\PY@tc##1{\textcolor[rgb]{0.00,0.63,0.00}{##1}}}
\expandafter\def\csname PY@tok@gr\endcsname{\def\PY@tc##1{\textcolor[rgb]{1.00,0.00,0.00}{##1}}}
\expandafter\def\csname PY@tok@ge\endcsname{\let\PY@it=\textit}
\expandafter\def\csname PY@tok@gs\endcsname{\let\PY@bf=\textbf}
\expandafter\def\csname PY@tok@gp\endcsname{\let\PY@bf=\textbf\def\PY@tc##1{\textcolor[rgb]{0.00,0.00,0.50}{##1}}}
\expandafter\def\csname PY@tok@go\endcsname{\def\PY@tc##1{\textcolor[rgb]{0.53,0.53,0.53}{##1}}}
\expandafter\def\csname PY@tok@gt\endcsname{\def\PY@tc##1{\textcolor[rgb]{0.00,0.27,0.87}{##1}}}
\expandafter\def\csname PY@tok@err\endcsname{\def\PY@bc##1{\setlength{\fboxsep}{0pt}\fcolorbox[rgb]{1.00,0.00,0.00}{1,1,1}{\strut ##1}}}
\expandafter\def\csname PY@tok@kc\endcsname{\let\PY@bf=\textbf\def\PY@tc##1{\textcolor[rgb]{0.00,0.50,0.00}{##1}}}
\expandafter\def\csname PY@tok@kd\endcsname{\let\PY@bf=\textbf\def\PY@tc##1{\textcolor[rgb]{0.00,0.50,0.00}{##1}}}
\expandafter\def\csname PY@tok@kn\endcsname{\let\PY@bf=\textbf\def\PY@tc##1{\textcolor[rgb]{0.00,0.50,0.00}{##1}}}
\expandafter\def\csname PY@tok@kr\endcsname{\let\PY@bf=\textbf\def\PY@tc##1{\textcolor[rgb]{0.00,0.50,0.00}{##1}}}
\expandafter\def\csname PY@tok@bp\endcsname{\def\PY@tc##1{\textcolor[rgb]{0.00,0.50,0.00}{##1}}}
\expandafter\def\csname PY@tok@fm\endcsname{\def\PY@tc##1{\textcolor[rgb]{0.00,0.00,1.00}{##1}}}
\expandafter\def\csname PY@tok@vc\endcsname{\def\PY@tc##1{\textcolor[rgb]{0.10,0.09,0.49}{##1}}}
\expandafter\def\csname PY@tok@vg\endcsname{\def\PY@tc##1{\textcolor[rgb]{0.10,0.09,0.49}{##1}}}
\expandafter\def\csname PY@tok@vi\endcsname{\def\PY@tc##1{\textcolor[rgb]{0.10,0.09,0.49}{##1}}}
\expandafter\def\csname PY@tok@vm\endcsname{\def\PY@tc##1{\textcolor[rgb]{0.10,0.09,0.49}{##1}}}
\expandafter\def\csname PY@tok@sa\endcsname{\def\PY@tc##1{\textcolor[rgb]{0.73,0.13,0.13}{##1}}}
\expandafter\def\csname PY@tok@sb\endcsname{\def\PY@tc##1{\textcolor[rgb]{0.73,0.13,0.13}{##1}}}
\expandafter\def\csname PY@tok@sc\endcsname{\def\PY@tc##1{\textcolor[rgb]{0.73,0.13,0.13}{##1}}}
\expandafter\def\csname PY@tok@dl\endcsname{\def\PY@tc##1{\textcolor[rgb]{0.73,0.13,0.13}{##1}}}
\expandafter\def\csname PY@tok@s2\endcsname{\def\PY@tc##1{\textcolor[rgb]{0.73,0.13,0.13}{##1}}}
\expandafter\def\csname PY@tok@sh\endcsname{\def\PY@tc##1{\textcolor[rgb]{0.73,0.13,0.13}{##1}}}
\expandafter\def\csname PY@tok@s1\endcsname{\def\PY@tc##1{\textcolor[rgb]{0.73,0.13,0.13}{##1}}}
\expandafter\def\csname PY@tok@mb\endcsname{\def\PY@tc##1{\textcolor[rgb]{0.40,0.40,0.40}{##1}}}
\expandafter\def\csname PY@tok@mf\endcsname{\def\PY@tc##1{\textcolor[rgb]{0.40,0.40,0.40}{##1}}}
\expandafter\def\csname PY@tok@mh\endcsname{\def\PY@tc##1{\textcolor[rgb]{0.40,0.40,0.40}{##1}}}
\expandafter\def\csname PY@tok@mi\endcsname{\def\PY@tc##1{\textcolor[rgb]{0.40,0.40,0.40}{##1}}}
\expandafter\def\csname PY@tok@il\endcsname{\def\PY@tc##1{\textcolor[rgb]{0.40,0.40,0.40}{##1}}}
\expandafter\def\csname PY@tok@mo\endcsname{\def\PY@tc##1{\textcolor[rgb]{0.40,0.40,0.40}{##1}}}
\expandafter\def\csname PY@tok@ch\endcsname{\let\PY@it=\textit\def\PY@tc##1{\textcolor[rgb]{0.25,0.50,0.50}{##1}}}
\expandafter\def\csname PY@tok@cm\endcsname{\let\PY@it=\textit\def\PY@tc##1{\textcolor[rgb]{0.25,0.50,0.50}{##1}}}
\expandafter\def\csname PY@tok@cpf\endcsname{\let\PY@it=\textit\def\PY@tc##1{\textcolor[rgb]{0.25,0.50,0.50}{##1}}}
\expandafter\def\csname PY@tok@c1\endcsname{\let\PY@it=\textit\def\PY@tc##1{\textcolor[rgb]{0.25,0.50,0.50}{##1}}}
\expandafter\def\csname PY@tok@cs\endcsname{\let\PY@it=\textit\def\PY@tc##1{\textcolor[rgb]{0.25,0.50,0.50}{##1}}}

\def\PYZbs{\char`\\}
\def\PYZus{\char`\_}
\def\PYZob{\char`\{}
\def\PYZcb{\char`\}}
\def\PYZca{\char`\^}
\def\PYZam{\char`\&}
\def\PYZlt{\char`\<}
\def\PYZgt{\char`\>}
\def\PYZsh{\char`\#}
\def\PYZpc{\char`\%}
\def\PYZdl{\char`\$}
\def\PYZhy{\char`\-}
\def\PYZsq{\char`\'}
\def\PYZdq{\char`\"}
\def\PYZti{\char`\~}
% for compatibility with earlier versions
\def\PYZat{@}
\def\PYZlb{[}
\def\PYZrb{]}
\makeatother


    % Exact colors from NB
    \definecolor{incolor}{rgb}{0.0, 0.0, 0.5}
    \definecolor{outcolor}{rgb}{0.545, 0.0, 0.0}



    
    % Prevent overflowing lines due to hard-to-break entities
    \sloppy 
    % Setup hyperref package
    \hypersetup{
      breaklinks=true,  % so long urls are correctly broken across lines
      colorlinks=true,
      urlcolor=urlcolor,
      linkcolor=linkcolor,
      citecolor=citecolor,
      }
    % Slightly bigger margins than the latex defaults
    
    \geometry{verbose,tmargin=1in,bmargin=1in,lmargin=1in,rmargin=1in}
    
    

    \begin{document}
    
    
    \maketitle
    
    

    
    Analyseur syntaxique et génération de code du langage facile. Le langage
facile est la suite du langage abordé en cours et consiste à implémenter
un langage simple dénué de fonctions

\hypertarget{avant-propos}{%
\section{Avant propos}\label{avant-propos}}

La structure des dossiers a été modifiée pour faciliter le développement
et l'évaluation. A la racine du projet on peut trouver:

\begin{itemize}
\item
  \texttt{build.sh}: le script permettant de compiler le programme
  \texttt{facile} et puis distribuer le binaire dans \texttt{dist/}.
  Mode d'emploi \texttt{sh\ build.sh}
\item
  \texttt{test.sh}: le script permettant d'effectuer un test à la fois
  les composants dans \texttt{/test}. Mode d'emploi
  \texttt{sh\ test.sh\ path/to/test/folder}
\end{itemize}

Le code CIL se trouve dans chaque sous-dossier dans \texttt{test}.

Le code source se trouve, naturellement, dans le dossier \texttt{src}.

Les instructions pour l'extensions (enrichissement de la grammaire)
\texttt{if\ /\ while\ /\ foreach} sont rajoutées dans
\texttt{(facile.y)\ instruction:}

\begin{verbatim}
instruction: affectation | print | read | if-stmt | 
    while-stmt | foreach-stmt | loop-interuptor;
\end{verbatim}

    \hypertarget{instruction-if}{%
\section{\texorpdfstring{Instruction
\texttt{IF}:}{Instruction IF:}}\label{instruction-if}}

    \hypertarget{analyse-lexicale}{%
\subsection{Analyse lexicale:}\label{analyse-lexicale}}

On souhaite définir les lexèmes pour \texttt{if}, \texttt{elseif},
\texttt{else}, \texttt{endif}, ainsi que les opérateurs logiques. On
prend en compte également la précédence des opérateurs. Par exemple:
\texttt{a\ \&\&\ b\ \textgreater{}=\ c} \(\leftrightarrow\)
\texttt{a\ \&\&\ (b\ \textgreater{}=\ c)}

\hypertarget{facile.lex}{%
\subsubsection{\texorpdfstring{\texttt{facile.lex}}{facile.lex}}\label{facile.lex}}

\begin{verbatim}
%{
/* Exercice 1: if elseif else */ 
%}

if    return TOK_IF;
then  return TOK_THEN;
end   return TOK_END;
endif return TOK_ENDIF;

elseif return TOK_ELSEIF;
else   return TOK_ELSE;

or    return TOK_BOOL_OR;
and   return TOK_BOOL_AND;
not   return TOK_BOOL_NOT;
false return TOK_BOOL_FALSE;
true  return TOK_BOOL_TRUE;
"="   return TOK_BOOL_EQ;
"#"   return TOK_BOOL_NEQ;
">="  return TOK_BOOL_GTE;
">"   return TOK_BOOL_GT;
"<="  return TOK_BOOL_LTE;
"<"   return TOK_BOOL_LT;
\end{verbatim}

\hypertarget{facile.y}{%
\subsubsection{\texorpdfstring{\texttt{facile.y}}{facile.y}}\label{facile.y}}

\begin{verbatim}
/* Logical OP */
%token TOK_BOOL_OR;
%token TOK_BOOL_AND;
%token TOK_BOOL_EQ;
%token TOK_BOOL_NEQ;
%token TOK_BOOL_GTE;
%token TOK_BOOL_GT;
%token TOK_BOOL_LTE;
%token TOK_BOOL_LT; 
%token TOK_BOOL_NOT;
%token TOK_BOOL_TRUE;
%token TOK_BOOL_FALSE;

/* Operator precedence */
%left TOK_BOOL_OR;
%left TOK_BOOL_AND;
%left TOK_BOOL_NOT;
%left TOK_BOOL_EQ TOK_BOOL_GT TOK_BOOL_GTE TOK_BOOL_LT TOK_BOOL_LTE TOK_BOOL_NEQ;

/* If ElseIf Else */
%token TOK_IF
%token TOK_THEN;
%token<string> TOK_END;
%token<string> TOK_ENDIF;

%token TOK_ELSEIF;
%token TOK_ELSE;
\end{verbatim}

    \hypertarget{analyse-syntaxique}{%
\subsection{Analyse syntaxique:}\label{analyse-syntaxique}}

En utilisant le diagramme de Conway, on peut traduire rapidement en
règles grammatiques.

\begin{itemize}
\tightlist
\item
  Exemple pour \textbf{\texttt{boolean\_expr}}
\end{itemize}

\begin{verbatim}
boolean_expr:
    TOK_OPEN_PARENTHESIS boolean_expr TOK_CLOSE_PARENTHESIS
    {
        $$ = g_node_new("boolexpr"); <- le nom sera util pour écrire le code CIL
        g_node_append($$, $2);
    }
|
    TOK_BOOL_TRUE
    {
        $$ = g_node_new("boolTrue"); <- Cas particulier 
    }
|
    TOK_BOOL_FALSE
    {
        $$ = g_node_new("boolFalse"); <- Cas particulier
    }
|
...
\end{verbatim}

\begin{itemize}
\tightlist
\item
  Exemple pour \textbf{\texttt{if-statement}}
\end{itemize}

\begin{verbatim}
if-stmt: 
    TOK_IF boolean_expr TOK_THEN code elseif else endif
    {
        $$ = g_node_new("if");
    g_node_append($$, $2);
    g_node_append($$, $4);
        g_node_append($$, $5);
    g_node_append($$, $6);
    g_node_append($$, $7);
    }
;

endif: <- On peut aussi termier simplement avec 'end' 
    TOK_END{$$ = g_node_new("endif");} | TOK_ENDIF{$$ = g_node_new("endif");}
;
\end{verbatim}

    \begin{itemize}
\tightlist
\item
  Exemple pour \textbf{\texttt{elseif\ /\ else}}
\end{itemize}

\begin{verbatim}
elseif: 
    // On peut avoir autant de elseif qu'on veut
    elseif TOK_ELSEIF boolean_expr TOK_THEN code elseif
    {
        $$ = g_node_new("elseif");
        g_node_append($$, $1);
        g_node_append($$, $3);
        g_node_append($$, $5);
        g_node_append($$, $6);
    }
|
    %empty
    {
        $$ = g_node_new("");
    }
;

else:
    TOK_ELSE code
    {
        $$ = g_node_new("else");
        g_node_append($$, $2);
    }
|
    // On peut ne pas avoir else
    %empty
    {
        $$ = g_node_new("");
    }
;
\end{verbatim}

    \hypertarget{guxe9nuxe9ration-de-code-cil}{%
\subsection{\texorpdfstring{Génération de code
\texttt{CIL}}{Génération de code CIL}}\label{guxe9nuxe9ration-de-code-cil}}

Le code généré n'est pas forcément le plus factorisé, mais il fait le
travail correctement. Pour le débogage, j'ai utilisé
\href{https://sharplab.io/}{SharpLab} pour apprendre et tester les codes
CIL.

    \hypertarget{expression-booluxe9enne}{%
\subsubsection{Expression booléenne}\label{expression-booluxe9enne}}

Pour le plupart des cas, les instructions de CIL suffisent amplement. Je
vais vous présenter les cas ou il faut chercher un peu plus loin.

   \begin{Verbatim}[commandchars=\\\{\},fontsize=\scriptsize]
{\color{incolor}In [{\color{incolor} }]:} \PY{c+cm}{/* Handle boolean\PYZus{}expr */}
        \PY{k}{else} \PY{n+nf}{if} \PY{p}{(}\PY{n}{node}\PY{o}{\PYZhy{}}\PY{o}{\PYZgt{}}\PY{n}{data} \PY{o}{=}\PY{o}{=} \PY{l+s}{\PYZdq{}}\PY{l+s}{boolTrue}\PY{l+s}{\PYZdq{}}\PY{p}{)}\PY{p}{\PYZob{}}
            \PY{n}{fprintf}\PY{p}{(}\PY{n}{stream}\PY{p}{,} \PY{l+s}{\PYZdq{}}\PY{l+s}{ ldc.i4.1}\PY{l+s+se}{\PYZbs{}n}\PY{l+s}{\PYZdq{}}\PY{p}{)}\PY{p}{;}
        \PY{p}{\PYZcb{}}
        
        \PY{k}{else} \PY{n+nf}{if} \PY{p}{(}\PY{n}{node}\PY{o}{\PYZhy{}}\PY{o}{\PYZgt{}}\PY{n}{data} \PY{o}{=}\PY{o}{=} \PY{l+s}{\PYZdq{}}\PY{l+s}{boolFalse}\PY{l+s}{\PYZdq{}}\PY{p}{)}\PY{p}{\PYZob{}}
            \PY{n}{fprintf}\PY{p}{(}\PY{n}{stream}\PY{p}{,} \PY{l+s}{\PYZdq{}}\PY{l+s}{ ldc.i4.0}\PY{l+s+se}{\PYZbs{}n}\PY{l+s}{\PYZdq{}}\PY{p}{)}\PY{p}{;}
        \PY{p}{\PYZcb{}}
        
        \PY{k}{else} \PY{n+nf}{if}\PY{p}{(}\PY{n}{node}\PY{o}{\PYZhy{}}\PY{o}{\PYZgt{}}\PY{n}{data} \PY{o}{=}\PY{o}{=} \PY{l+s}{\PYZdq{}}\PY{l+s}{not}\PY{l+s}{\PYZdq{}}\PY{p}{)}\PY{p}{\PYZob{}}
            \PY{n}{produce\PYZus{}code}\PY{p}{(}\PY{n}{g\PYZus{}node\PYZus{}nth\PYZus{}child}\PY{p}{(}\PY{n}{node}\PY{p}{,} \PY{l+m+mi}{0}\PY{p}{)}\PY{p}{)}\PY{p}{;} \PY{c+c1}{// \PYZsq{}a\PYZsq{} la valeur dans le stack}
            \PY{n}{fprintf}\PY{p}{(}\PY{n}{stream}\PY{p}{,} \PY{l+s}{\PYZdq{}}\PY{l+s}{    ldc.i4.0}\PY{l+s+se}{\PYZbs{}n}\PY{l+s}{ceq}\PY{l+s+se}{\PYZbs{}n}\PY{l+s}{\PYZdq{}}\PY{p}{)}\PY{p}{;} \PY{c+c1}{// 1 si a == 0  sinon 0}
        \PY{p}{\PYZcb{}}
        
        \PY{k}{else} \PY{n+nf}{if}\PY{p}{(}\PY{n}{node}\PY{o}{\PYZhy{}}\PY{o}{\PYZgt{}}\PY{n}{data} \PY{o}{=}\PY{o}{=} \PY{l+s}{\PYZdq{}}\PY{l+s}{neq}\PY{l+s}{\PYZdq{}}\PY{p}{)}\PY{p}{\PYZob{}}
            \PY{n}{produce\PYZus{}code}\PY{p}{(}\PY{n}{g\PYZus{}node\PYZus{}nth\PYZus{}child}\PY{p}{(}\PY{n}{node}\PY{p}{,} \PY{l+m+mi}{0}\PY{p}{)}\PY{p}{)}\PY{p}{;}
            \PY{n}{produce\PYZus{}code}\PY{p}{(}\PY{n}{g\PYZus{}node\PYZus{}nth\PYZus{}child}\PY{p}{(}\PY{n}{node}\PY{p}{,} \PY{l+m+mi}{1}\PY{p}{)}\PY{p}{)}\PY{p}{;}
            \PY{n}{fprintf}\PY{p}{(}\PY{n}{stream}\PY{p}{,} \PY{l+s}{\PYZdq{}}\PY{l+s}{    ceq}\PY{l+s+se}{\PYZbs{}n}\PY{l+s}{ldc.i4.0}\PY{l+s+se}{\PYZbs{}n}\PY{l+s}{ceq}\PY{l+s+se}{\PYZbs{}n}\PY{l+s}{\PYZdq{}}\PY{p}{)}\PY{p}{;} \PY{c+c1}{// litterally \PYZdq{}equal + not\PYZdq{}}
        \PY{p}{\PYZcb{}}
        
        \PY{k}{else} \PY{n+nf}{if}\PY{p}{(}\PY{n}{node}\PY{o}{\PYZhy{}}\PY{o}{\PYZgt{}}\PY{n}{data} \PY{o}{=}\PY{o}{=} \PY{l+s}{\PYZdq{}}\PY{l+s}{gte}\PY{l+s}{\PYZdq{}}\PY{p}{)}\PY{p}{\PYZob{}}
            \PY{n}{produce\PYZus{}code}\PY{p}{(}\PY{n}{g\PYZus{}node\PYZus{}nth\PYZus{}child}\PY{p}{(}\PY{n}{node}\PY{p}{,} \PY{l+m+mi}{0}\PY{p}{)}\PY{p}{)}\PY{p}{;}
            \PY{n}{produce\PYZus{}code}\PY{p}{(}\PY{n}{g\PYZus{}node\PYZus{}nth\PYZus{}child}\PY{p}{(}\PY{n}{node}\PY{p}{,} \PY{l+m+mi}{1}\PY{p}{)}\PY{p}{)}\PY{p}{;}
            \PY{n}{fprintf}\PY{p}{(}\PY{n}{stream}\PY{p}{,} \PY{l+s}{\PYZdq{}}\PY{l+s}{    clt}\PY{l+s+se}{\PYZbs{}n}\PY{l+s}{ldc.i4.0}\PY{l+s+se}{\PYZbs{}n}\PY{l+s}{ceq}\PY{l+s+se}{\PYZbs{}n}\PY{l+s}{\PYZdq{}}\PY{p}{)}\PY{p}{;} \PY{c+c1}{// litt. \PYZdq{}not lesser than\PYZdq{}}
        \PY{p}{\PYZcb{}}
        
        \PY{k}{else} \PY{n+nf}{if}\PY{p}{(}\PY{n}{node}\PY{o}{\PYZhy{}}\PY{o}{\PYZgt{}}\PY{n}{data} \PY{o}{=}\PY{o}{=} \PY{l+s}{\PYZdq{}}\PY{l+s}{lte}\PY{l+s}{\PYZdq{}}\PY{p}{)}\PY{p}{\PYZob{}}
            \PY{n}{produce\PYZus{}code}\PY{p}{(}\PY{n}{g\PYZus{}node\PYZus{}nth\PYZus{}child}\PY{p}{(}\PY{n}{node}\PY{p}{,} \PY{l+m+mi}{0}\PY{p}{)}\PY{p}{)}\PY{p}{;}
            \PY{n}{produce\PYZus{}code}\PY{p}{(}\PY{n}{g\PYZus{}node\PYZus{}nth\PYZus{}child}\PY{p}{(}\PY{n}{node}\PY{p}{,} \PY{l+m+mi}{1}\PY{p}{)}\PY{p}{)}\PY{p}{;}
            \PY{n}{fprintf}\PY{p}{(}\PY{n}{stream}\PY{p}{,} \PY{l+s}{\PYZdq{}}\PY{l+s}{    cgt}\PY{l+s+se}{\PYZbs{}n}\PY{l+s}{ldc.i4.0}\PY{l+s+se}{\PYZbs{}n}\PY{l+s}{ceq}\PY{l+s+se}{\PYZbs{}n}\PY{l+s}{\PYZdq{}}\PY{p}{)}\PY{p}{;} \PY{c+c1}{// litt. \PYZdq{}not greater than\PYZdq{}}
        \PY{p}{\PYZcb{}}
\end{Verbatim}

    \hypertarget{if-elseif-else}{%
\subsubsection{\texorpdfstring{\texttt{If\ /\ ElseIf\ /\ Else}}{If / ElseIf / Else}}\label{if-elseif-else}}

On veut simplement sauter à un autre endroit (offset) si la condition
n'est pas vérifier. L'instruction
\texttt{brfalse.s\ \textless{}target\textgreater{}} permet de faire
exactement ça. Pour aider, j'ai déclaré différents types de offset au
début comme variable globale, qui sera incrémenté à chaque exécution de
\texttt{produce\_code()}.

En outre chaque instruction CIL est marqué avec un offset par défaut et
qu'on peut le renommer:

\begin{verbatim}
IL_0000: <codeCIL>
JUMP_POINT_0: <codeCIL>
\end{verbatim}

   \begin{Verbatim}[commandchars=\\\{\},fontsize=\scriptsize]
{\color{incolor}In [{\color{incolor} }]:} \PY{k+kt}{int} \PY{n+nf}{yylex}\PY{p}{(}\PY{p}{)}\PY{p}{;}
        \PY{k+kt}{int} \PY{n+nf}{yyerror}\PY{p}{(}\PY{k+kt}{char} \PY{o}{*}\PY{n}{msg}\PY{p}{)}\PY{p}{;}
        
        \PY{n}{GHashTable} \PY{o}{*}\PY{n}{table}\PY{p}{;}
        
        \PY{c+c1}{// Offset \PYZhy{} Useful for branching}
        \PY{k+kt}{unsigned} \PY{k+kt}{int} \PY{n}{offset} \PY{o}{=} \PY{l+m+mi}{0}\PY{p}{;} \PY{c+c1}{// +1 everytime}
        \PY{k+kt}{unsigned} \PY{k+kt}{int} \PY{n}{loop\PYZus{}offset} \PY{o}{=} \PY{l+m+mi}{0}\PY{p}{;} \PY{c+c1}{// +1 when loop}
        
        \PY{k+kt}{FILE} \PY{o}{*}\PY{n}{stream}\PY{p}{;}
        \PY{k+kt}{char} \PY{o}{*}\PY{n}{module\PYZus{}name}\PY{p}{;}
        \PY{k+kt}{unsigned} \PY{k+kt}{int} \PY{n}{max\PYZus{}stack} \PY{o}{=} \PY{l+m+mi}{10}\PY{p}{;}
        
        \PY{k}{extern} \PY{k+kt}{void} \PY{n+nf}{begin\PYZus{}code}\PY{p}{(}\PY{p}{)}\PY{p}{;}
        \PY{k}{extern} \PY{k+kt}{void} \PY{n+nf}{end\PYZus{}code}\PY{p}{(}\PY{p}{)}\PY{p}{;}
        \PY{k}{extern} \PY{k+kt}{void} \PY{n+nf}{produce\PYZus{}code}\PY{p}{(}\PY{n}{GNode} \PY{o}{*} \PY{n}{node}\PY{p}{)}\PY{p}{;}
\end{Verbatim}

   \begin{Verbatim}[commandchars=\\\{\},fontsize=\scriptsize]
{\color{incolor}In [{\color{incolor} }]:} \PY{c+cm}{/* If statement */}
        \PY{k}{else} \PY{n+nf}{if}\PY{p}{(}\PY{n}{node}\PY{o}{\PYZhy{}}\PY{o}{\PYZgt{}}\PY{n}{data} \PY{o}{=}\PY{o}{=} \PY{l+s}{\PYZdq{}}\PY{l+s}{if}\PY{l+s}{\PYZdq{}}\PY{p}{)}\PY{p}{\PYZob{}}
            \PY{n}{produce\PYZus{}code}\PY{p}{(}\PY{n}{g\PYZus{}node\PYZus{}nth\PYZus{}child}\PY{p}{(}\PY{n}{node}\PY{p}{,} \PY{l+m+mi}{0}\PY{p}{)}\PY{p}{)}\PY{p}{;} \PY{c+c1}{// boolean\PYZus{}expr}
        
            \PY{c+c1}{// Mark the jump address}
            \PY{n}{guint} \PY{n}{endSbl} \PY{o}{=} \PY{n}{offset}\PY{p}{;}
            \PY{n}{fprintf}\PY{p}{(}\PY{n}{stream}\PY{p}{,} \PY{l+s}{\PYZdq{}}\PY{l+s}{   brfalse.s IF\PYZus{}\PYZpc{}d}\PY{l+s+se}{\PYZbs{}n}\PY{l+s+se}{\PYZbs{}n}\PY{l+s}{\PYZdq{}}\PY{p}{,} \PY{n}{endSbl}\PY{p}{)}\PY{p}{;}
        
            \PY{n}{fprintf}\PY{p}{(}\PY{n}{stream}\PY{p}{,} \PY{l+s}{\PYZdq{}}\PY{l+s}{   nop}\PY{l+s+se}{\PYZbs{}n}\PY{l+s}{\PYZdq{}}\PY{p}{)}\PY{p}{;}
            \PY{n}{produce\PYZus{}code}\PY{p}{(}\PY{n}{g\PYZus{}node\PYZus{}nth\PYZus{}child}\PY{p}{(}\PY{n}{node}\PY{p}{,} \PY{l+m+mi}{1}\PY{p}{)}\PY{p}{)}\PY{p}{;} \PY{c+c1}{// code wrapped with nop for catching}
            \PY{n}{fprintf}\PY{p}{(}\PY{n}{stream}\PY{p}{,} \PY{l+s}{\PYZdq{}}\PY{l+s}{   nop}\PY{l+s+se}{\PYZbs{}n}\PY{l+s}{\PYZdq{}}\PY{p}{)}\PY{p}{;}
        
            \PY{n}{fprintf}\PY{p}{(}\PY{n}{stream}\PY{p}{,} \PY{l+s}{\PYZdq{}}\PY{l+s}{   nop}\PY{l+s+se}{\PYZbs{}n}\PY{l+s}{\PYZdq{}}\PY{p}{)}\PY{p}{;}
            \PY{n}{fprintf}\PY{p}{(}\PY{n}{stream}\PY{p}{,} \PY{l+s}{\PYZdq{}}\PY{l+s}{   IF\PYZus{}\PYZpc{}d:}\PY{l+s}{\PYZdq{}}\PY{p}{,} \PY{n}{endSbl}\PY{p}{)}\PY{p}{;} \PY{c+c1}{// end of code, mark jump point}
        
            \PY{n}{produce\PYZus{}code}\PY{p}{(}\PY{n}{g\PYZus{}node\PYZus{}nth\PYZus{}child}\PY{p}{(}\PY{n}{node}\PY{p}{,} \PY{l+m+mi}{2}\PY{p}{)}\PY{p}{)}\PY{p}{;} \PY{c+c1}{// elseif}
            \PY{n}{produce\PYZus{}code}\PY{p}{(}\PY{n}{g\PYZus{}node\PYZus{}nth\PYZus{}child}\PY{p}{(}\PY{n}{node}\PY{p}{,} \PY{l+m+mi}{3}\PY{p}{)}\PY{p}{)}\PY{p}{;} \PY{c+c1}{// else}
            \PY{n}{produce\PYZus{}code}\PY{p}{(}\PY{n}{g\PYZus{}node\PYZus{}nth\PYZus{}child}\PY{p}{(}\PY{n}{node}\PY{p}{,} \PY{l+m+mi}{4}\PY{p}{)}\PY{p}{)}\PY{p}{;} \PY{c+c1}{// endif}
        \PY{p}{\PYZcb{}}
\end{Verbatim}

    \texttt{elseif} reprend la même code CIL que \texttt{if}

   \begin{Verbatim}[commandchars=\\\{\},fontsize=\scriptsize]
{\color{incolor}In [{\color{incolor} }]:} \PY{c+cm}{/*ElseIf statement*/}
        \PY{k}{else} \PY{n+nf}{if}\PY{p}{(}\PY{n}{node}\PY{o}{\PYZhy{}}\PY{o}{\PYZgt{}}\PY{n}{data} \PY{o}{=}\PY{o}{=} \PY{l+s}{\PYZdq{}}\PY{l+s}{elseif}\PY{l+s}{\PYZdq{}}\PY{p}{)}\PY{p}{\PYZob{}}
            \PY{n}{produce\PYZus{}code}\PY{p}{(}\PY{n}{g\PYZus{}node\PYZus{}nth\PYZus{}child}\PY{p}{(}\PY{n}{node}\PY{p}{,} \PY{l+m+mi}{0}\PY{p}{)}\PY{p}{)}\PY{p}{;} \PY{c+c1}{// elseif}
            \PY{n}{produce\PYZus{}code}\PY{p}{(}\PY{n}{g\PYZus{}node\PYZus{}nth\PYZus{}child}\PY{p}{(}\PY{n}{node}\PY{p}{,} \PY{l+m+mi}{1}\PY{p}{)}\PY{p}{)}\PY{p}{;} \PY{c+c1}{// boolean\PYZus{}expr}
        
            \PY{c+c1}{// Mark the jump address}
            \PY{n}{guint} \PY{n}{endSbl} \PY{o}{=} \PY{n}{offset}\PY{p}{;}
            \PY{n}{fprintf}\PY{p}{(}\PY{n}{stream}\PY{p}{,} \PY{l+s}{\PYZdq{}}\PY{l+s}{   brfalse.s ELSEIF\PYZus{}\PYZpc{}d}\PY{l+s+se}{\PYZbs{}n}\PY{l+s+se}{\PYZbs{}n}\PY{l+s}{\PYZdq{}}\PY{p}{,} \PY{n}{endSbl}\PY{p}{)}\PY{p}{;}
        
            \PY{n}{fprintf}\PY{p}{(}\PY{n}{stream}\PY{p}{,} \PY{l+s}{\PYZdq{}}\PY{l+s}{   nop}\PY{l+s+se}{\PYZbs{}n}\PY{l+s}{\PYZdq{}}\PY{p}{)}\PY{p}{;}
            \PY{n}{produce\PYZus{}code}\PY{p}{(}\PY{n}{g\PYZus{}node\PYZus{}nth\PYZus{}child}\PY{p}{(}\PY{n}{node}\PY{p}{,} \PY{l+m+mi}{2}\PY{p}{)}\PY{p}{)}\PY{p}{;} \PY{c+c1}{// code}
            \PY{n}{fprintf}\PY{p}{(}\PY{n}{stream}\PY{p}{,} \PY{l+s}{\PYZdq{}}\PY{l+s}{   nop}\PY{l+s+se}{\PYZbs{}n}\PY{l+s}{\PYZdq{}}\PY{p}{)}\PY{p}{;}
        
            \PY{n}{fprintf}\PY{p}{(}\PY{n}{stream}\PY{p}{,} \PY{l+s}{\PYZdq{}}\PY{l+s}{   nop}\PY{l+s+se}{\PYZbs{}n}\PY{l+s}{\PYZdq{}}\PY{p}{)}\PY{p}{;}
            \PY{n}{fprintf}\PY{p}{(}\PY{n}{stream}\PY{p}{,} \PY{l+s}{\PYZdq{}}\PY{l+s}{   br.s IL\PYZus{}LAST}\PY{l+s+se}{\PYZbs{}n}\PY{l+s+se}{\PYZbs{}n}\PY{l+s}{\PYZdq{}}\PY{p}{)}\PY{p}{;}
            \PY{n}{fprintf}\PY{p}{(}\PY{n}{stream}\PY{p}{,} \PY{l+s}{\PYZdq{}}\PY{l+s}{   ELSEIF\PYZus{}\PYZpc{}d: }\PY{l+s}{\PYZdq{}}\PY{p}{,} \PY{n}{endSbl}\PY{p}{)}\PY{p}{;} \PY{c+c1}{// end of code, mark jump point}
        
            \PY{n}{produce\PYZus{}code}\PY{p}{(}\PY{n}{g\PYZus{}node\PYZus{}nth\PYZus{}child}\PY{p}{(}\PY{n}{node}\PY{p}{,} \PY{l+m+mi}{3}\PY{p}{)}\PY{p}{)}\PY{p}{;} \PY{c+c1}{// elseif}
        \PY{p}{\PYZcb{}}
        
        \PY{c+cm}{/* Else */}
        \PY{k}{else} \PY{n+nf}{if}\PY{p}{(}\PY{n}{node}\PY{o}{\PYZhy{}}\PY{o}{\PYZgt{}}\PY{n}{data} \PY{o}{=}\PY{o}{=} \PY{l+s}{\PYZdq{}}\PY{l+s}{else}\PY{l+s}{\PYZdq{}}\PY{p}{)}\PY{p}{\PYZob{}}
            \PY{n}{fprintf}\PY{p}{(}\PY{n}{stream}\PY{p}{,} \PY{l+s}{\PYZdq{}}\PY{l+s}{   nop}\PY{l+s+se}{\PYZbs{}n}\PY{l+s}{\PYZdq{}}\PY{p}{)}\PY{p}{;}
            \PY{n}{produce\PYZus{}code}\PY{p}{(}\PY{n}{g\PYZus{}node\PYZus{}nth\PYZus{}child}\PY{p}{(}\PY{n}{node}\PY{p}{,} \PY{l+m+mi}{0}\PY{p}{)}\PY{p}{)}\PY{p}{;} \PY{c+c1}{// code}
            \PY{n}{fprintf}\PY{p}{(}\PY{n}{stream}\PY{p}{,} \PY{l+s}{\PYZdq{}}\PY{l+s}{   nop}\PY{l+s+se}{\PYZbs{}n}\PY{l+s}{\PYZdq{}}\PY{p}{)}\PY{p}{;}
        \PY{p}{\PYZcb{}}
\end{Verbatim}

    \hypertarget{instruction-while-foreach}{%
\section{\texorpdfstring{Instruction
\texttt{WHILE\ /\ FOREACH}}{Instruction WHILE / FOREACH}}\label{instruction-while-foreach}}

Dans cette partie, on va traiter les boucles. \texttt{foreach} n'est
qu'un \texttt{while} spécial avec un compteur.

On gère également les interupteurs \texttt{continue} et \texttt{break}

    \hypertarget{analyse-lexicale}{%
\subsection{Analyse lexicale:}\label{analyse-lexicale}}

On souhaite définir les lexèmes pour \texttt{while}, \texttt{foreach}.

\hypertarget{facile.lex}{%
\subsubsection{\texorpdfstring{\texttt{facile.lex}}{facile.lex}}\label{facile.lex}}

\begin{verbatim}
%{
/* Exercice 2: While */ 
%}
while     return TOK_WHILE;
do        return TOK_DO;
endwhile  return TOK_ENDWHILE;
".."      return TOK_ARR_TO;
continue  return TOK_CONTINUE;
break     return TOK_BREAK;

%{
/* Exercice 3: Foreach */ 
%}
foreach     return TOK_FOREACH;
in          return TOK_IN;
endforeach  return TOK_ENDFOREACH;
\end{verbatim}

\hypertarget{facile.y}{%
\subsubsection{\texorpdfstring{\texttt{facile.y}}{facile.y}}\label{facile.y}}

\begin{verbatim}
%type<node> while-stmt
%type<node> endwhile

%type<node> foreach-stmt
%type<node> endforeach
%type<node> loop-interuptor
\end{verbatim}

    \hypertarget{analyse-syntaxique}{%
\subsection{Analyse syntaxique:}\label{analyse-syntaxique}}

En utilisant le diagramme de Conway, on peut traduire rapidement en
règles grammatiques.

\begin{itemize}
\tightlist
\item
  Exemple pour \textbf{\texttt{loop-interuptor}}
\end{itemize}

\begin{verbatim}
loop-interuptor:
    TOK_CONTINUE TOK_SEMICOLON
    {
        $$ = g_node_new("skipItr");
    }
|
    TOK_BREAK TOK_SEMICOLON
    {
        $$ = g_node_new("breakLoop");
    }
;
\end{verbatim}

\begin{itemize}
\tightlist
\item
  Exemple pour \textbf{\texttt{while-stmt}}
\end{itemize}

\begin{verbatim}
while-stmt:
    TOK_WHILE boolean_expr TOK_DO code endwhile
    {
        $$ = g_node_new("while");
        g_node_append($$, $2);
        g_node_append($$, $4);
        g_node_append($$, $5);
    }
;

endwhile:
    TOK_END
    {
        $$ = g_node_new("endwhile");
    }
|
    TOK_ENDWHILE
    {
        $$ = g_node_new("endwhile");
    }
;
\end{verbatim}

\begin{itemize}
\tightlist
\item
  Exemple pour \textbf{\texttt{foreach-stmt}}
\end{itemize}

\begin{verbatim}
foreach-stmt:
    TOK_FOREACH ident TOK_IN expr TOK_ARR_TO expr TOK_DO code endforeach
    {
        $$ = g_node_new("foreach");
        g_node_append($$, $4);
        g_node_append($$, $6);
        g_node_append($$, $8);
        g_node_append($$, $9);
    }
;

endforeach:
    TOK_END
    {
        $$ = g_node_new("endforeach");
    }
|
    TOK_ENDFOREACH
    {
        $$ = g_node_new("endforeach");
    }
;
\end{verbatim}

    Un interrupteur peut être \texttt{continue} ou \texttt{break}. Ce sont
tous les deux des sauts vers un autre point dans le code.
\texttt{continue} saute vers la prochaine itération, marquée mar une
incrémentation du compteur puis une vérification de la condition. Au
contraire, \texttt{break} saute vers la fin de la boucle.

   \begin{Verbatim}[commandchars=\\\{\},fontsize=\scriptsize]
{\color{incolor}In [{\color{incolor} }]:} \PY{c+cm}{/* Loop interuptor */}
        \PY{k}{else} \PY{n+nf}{if}\PY{p}{(}\PY{n}{node}\PY{o}{\PYZhy{}}\PY{o}{\PYZgt{}}\PY{n}{data} \PY{o}{=}\PY{o}{=} \PY{l+s}{\PYZdq{}}\PY{l+s}{skipItr}\PY{l+s}{\PYZdq{}}\PY{p}{)}\PY{p}{\PYZob{}}
            \PY{n}{guint} \PY{n}{endSbl} \PY{o}{=} \PY{n}{loop\PYZus{}offset}\PY{p}{;}
            \PY{n}{fprintf}\PY{p}{(}\PY{n}{stream}\PY{p}{,} \PY{l+s}{\PYZdq{}}\PY{l+s}{   br.s LOOP\PYZus{}INCR\PYZus{}\PYZpc{}d}\PY{l+s+se}{\PYZbs{}n}\PY{l+s+se}{\PYZbs{}n}\PY{l+s}{\PYZdq{}}\PY{p}{,} \PY{n}{endSbl}\PY{p}{)}\PY{p}{;}
        \PY{p}{\PYZcb{}}
        
        \PY{k}{else} \PY{n+nf}{if}\PY{p}{(}\PY{n}{node}\PY{o}{\PYZhy{}}\PY{o}{\PYZgt{}}\PY{n}{data} \PY{o}{=}\PY{o}{=} \PY{l+s}{\PYZdq{}}\PY{l+s}{breakLoop}\PY{l+s}{\PYZdq{}}\PY{p}{)}\PY{p}{\PYZob{}}
            \PY{n}{guint} \PY{n}{endSbl} \PY{o}{=} \PY{n}{loop\PYZus{}offset}\PY{p}{;}
            \PY{n}{fprintf}\PY{p}{(}\PY{n}{stream}\PY{p}{,} \PY{l+s}{\PYZdq{}}\PY{l+s}{   br.s LOOP\PYZus{}END\PYZus{}\PYZpc{}d}\PY{l+s+se}{\PYZbs{}n}\PY{l+s+se}{\PYZbs{}n}\PY{l+s}{\PYZdq{}}\PY{p}{,} \PY{n}{endSbl}\PY{p}{)}\PY{p}{;}
        \PY{p}{\PYZcb{}}
\end{Verbatim}

   \begin{Verbatim}[commandchars=\\\{\},fontsize=\scriptsize]
{\color{incolor}In [{\color{incolor} }]:} \PY{c+cm}{/* While */}
        \PY{k}{else} \PY{n+nf}{if}\PY{p}{(}\PY{n}{node}\PY{o}{\PYZhy{}}\PY{o}{\PYZgt{}}\PY{n}{data} \PY{o}{=}\PY{o}{=} \PY{l+s}{\PYZdq{}}\PY{l+s}{while}\PY{l+s}{\PYZdq{}}\PY{p}{)}\PY{p}{\PYZob{}}
        
            \PY{n}{loop\PYZus{}offset}\PY{o}{+}\PY{o}{+}\PY{p}{;}
        
            \PY{c+c1}{// Branch out}
            \PY{n}{guint} \PY{n}{endSbl} \PY{o}{=} \PY{n}{loop\PYZus{}offset}\PY{p}{;}
            \PY{n}{fprintf}\PY{p}{(}\PY{n}{stream}\PY{p}{,} \PY{l+s}{\PYZdq{}}\PY{l+s}{   br.s LOOP\PYZus{}HEAD\PYZus{}\PYZpc{}d}\PY{l+s+se}{\PYZbs{}n}\PY{l+s}{\PYZdq{}}\PY{p}{,} \PY{n}{endSbl}\PY{p}{)}\PY{p}{;} \PY{c+c1}{// Init first iteration by}
        \PY{n}{jumping} \PY{n}{to} \PY{n}{head}
            \PY{n}{fprintf}\PY{p}{(}\PY{n}{stream}\PY{p}{,} \PY{l+s}{\PYZdq{}}\PY{l+s}{   // Start loop (head: LOOP\PYZus{}HEAD\PYZus{}\PYZpc{}d)}\PY{l+s+se}{\PYZbs{}n}\PY{l+s}{\PYZdq{}}\PY{p}{,} \PY{n}{endSbl}\PY{p}{)}\PY{p}{;}
            \PY{n}{fprintf}\PY{p}{(}\PY{n}{stream}\PY{p}{,} \PY{l+s}{\PYZdq{}}\PY{l+s}{   LOOP\PYZus{}START\PYZus{}\PYZpc{}d: nop}\PY{l+s+se}{\PYZbs{}n}\PY{l+s}{\PYZdq{}}\PY{p}{,} \PY{n}{endSbl}\PY{p}{)}\PY{p}{;} \PY{c+c1}{// Mark beginning}
        
            \PY{n}{fprintf}\PY{p}{(}\PY{n}{stream}\PY{p}{,} \PY{l+s}{\PYZdq{}}\PY{l+s}{   nop}\PY{l+s+se}{\PYZbs{}n}\PY{l+s}{\PYZdq{}}\PY{p}{)}\PY{p}{;} \PY{c+c1}{// un code sans nop c\PYZsq{}est comme un jour sans le soleil}
            \PY{n}{produce\PYZus{}code}\PY{p}{(}\PY{n}{g\PYZus{}node\PYZus{}nth\PYZus{}child}\PY{p}{(}\PY{n}{node}\PY{p}{,} \PY{l+m+mi}{1}\PY{p}{)}\PY{p}{)}\PY{p}{;} \PY{c+c1}{// code}
            \PY{n}{fprintf}\PY{p}{(}\PY{n}{stream}\PY{p}{,} \PY{l+s}{\PYZdq{}}\PY{l+s}{   nop}\PY{l+s+se}{\PYZbs{}n}\PY{l+s}{\PYZdq{}}\PY{p}{)}\PY{p}{;}
        
            \PY{c+c1}{// Non existing counter}
            \PY{n}{fprintf}\PY{p}{(}\PY{n}{stream}\PY{p}{,} \PY{l+s}{\PYZdq{}}\PY{l+s}{   LOOP\PYZus{}INCR\PYZus{}\PYZpc{}d: nop}\PY{l+s+se}{\PYZbs{}n}\PY{l+s}{\PYZdq{}}\PY{p}{,} \PY{n}{endSbl}\PY{p}{)}\PY{p}{;}
        
            \PY{n}{fprintf}\PY{p}{(}\PY{n}{stream}\PY{p}{,} \PY{l+s}{\PYZdq{}}\PY{l+s}{   LOOP\PYZus{}HEAD\PYZus{}\PYZpc{}d: }\PY{l+s}{\PYZdq{}}\PY{p}{,} \PY{n}{endSbl}\PY{p}{)}\PY{p}{;} \PY{c+c1}{// Mark head}
            \PY{n}{produce\PYZus{}code}\PY{p}{(}\PY{n}{g\PYZus{}node\PYZus{}nth\PYZus{}child}\PY{p}{(}\PY{n}{node}\PY{p}{,} \PY{l+m+mi}{0}\PY{p}{)}\PY{p}{)}\PY{p}{;} \PY{c+c1}{// boolean\PYZus{}expr}
            \PY{n}{fprintf}\PY{p}{(}\PY{n}{stream}\PY{p}{,} \PY{l+s}{\PYZdq{}}\PY{l+s}{   brtrue.s LOOP\PYZus{}START\PYZus{}\PYZpc{}d}\PY{l+s+se}{\PYZbs{}n}\PY{l+s}{\PYZdq{}}\PY{p}{,} \PY{n}{endSbl}\PY{p}{)}\PY{p}{;} \PY{c+c1}{// jump to beginning of loop}
        \PY{k}{if} \PY{n}{cond}
        
            \PY{n}{produce\PYZus{}code}\PY{p}{(}\PY{n}{g\PYZus{}node\PYZus{}nth\PYZus{}child}\PY{p}{(}\PY{n}{node}\PY{p}{,} \PY{l+m+mi}{2}\PY{p}{)}\PY{p}{)}\PY{p}{;} \PY{c+c1}{// endwhile}
            \PY{n}{fprintf}\PY{p}{(}\PY{n}{stream}\PY{p}{,} \PY{l+s}{\PYZdq{}}\PY{l+s}{   // End loop}\PY{l+s+se}{\PYZbs{}n}\PY{l+s}{\PYZdq{}}\PY{p}{)}\PY{p}{;}
            \PY{n}{fprintf}\PY{p}{(}\PY{n}{stream}\PY{p}{,} \PY{l+s}{\PYZdq{}}\PY{l+s}{   LOOP\PYZus{}END\PYZus{}\PYZpc{}d: nop}\PY{l+s+se}{\PYZbs{}n}\PY{l+s}{\PYZdq{}}\PY{p}{,} \PY{n}{endSbl}\PY{p}{)}\PY{p}{;}
        \PY{p}{\PYZcb{}}
\end{Verbatim}

   \begin{Verbatim}[commandchars=\\\{\},fontsize=\scriptsize]
{\color{incolor}In [{\color{incolor} }]:} \PY{c+cm}{/* Foreach */}
        \PY{k}{else} \PY{n+nf}{if}\PY{p}{(}\PY{n}{node}\PY{o}{\PYZhy{}}\PY{o}{\PYZgt{}}\PY{n}{data} \PY{o}{=}\PY{o}{=} \PY{l+s}{\PYZdq{}}\PY{l+s}{foreach}\PY{l+s}{\PYZdq{}}\PY{p}{)}\PY{p}{\PYZob{}}
        
            \PY{n}{loop\PYZus{}offset}\PY{o}{+}\PY{o}{+}\PY{p}{;}
            \PY{n}{guint} \PY{n}{endSbl} \PY{o}{=} \PY{n}{loop\PYZus{}offset}\PY{p}{;}
        
            \PY{c+c1}{// Initialise counter variable}
            \PY{n}{fprintf}\PY{p}{(}\PY{n}{stream}\PY{p}{,} \PY{l+s}{\PYZdq{}}\PY{l+s}{   nop}\PY{l+s+se}{\PYZbs{}n}\PY{l+s}{\PYZdq{}}\PY{p}{)}\PY{p}{;}
            \PY{n}{produce\PYZus{}code}\PY{p}{(}\PY{n}{g\PYZus{}node\PYZus{}nth\PYZus{}child}\PY{p}{(}\PY{n}{node}\PY{p}{,} \PY{l+m+mi}{0}\PY{p}{)}\PY{p}{)}\PY{p}{;} \PY{c+c1}{// expr}
            \PY{n}{fprintf}\PY{p}{(}\PY{n}{stream}\PY{p}{,} \PY{l+s}{\PYZdq{}}\PY{l+s}{   stloc.s \PYZpc{}d}\PY{l+s+se}{\PYZbs{}n}\PY{l+s}{\PYZdq{}}\PY{p}{,} \PY{n}{endSbl}\PY{p}{)}\PY{p}{;}
        
            \PY{c+c1}{// Branch off \PYZhy{} while like}
            \PY{n}{fprintf}\PY{p}{(}\PY{n}{stream}\PY{p}{,} \PY{l+s}{\PYZdq{}}\PY{l+s}{   br.s LOOP\PYZus{}HEAD\PYZus{}\PYZpc{}d}\PY{l+s+se}{\PYZbs{}n}\PY{l+s}{\PYZdq{}}\PY{p}{,} \PY{n}{endSbl}\PY{p}{)}\PY{p}{;} \PY{c+c1}{// Jump to head}
            \PY{n}{fprintf}\PY{p}{(}\PY{n}{stream}\PY{p}{,} \PY{l+s}{\PYZdq{}}\PY{l+s}{  // Start loop (head: LOOP\PYZus{}HEAD\PYZus{}\PYZpc{}d)}\PY{l+s+se}{\PYZbs{}n}\PY{l+s}{\PYZdq{}}\PY{p}{,} \PY{n}{endSbl}\PY{p}{)}\PY{p}{;}
        
            \PY{n}{fprintf}\PY{p}{(}\PY{n}{stream}\PY{p}{,} \PY{l+s}{\PYZdq{}}\PY{l+s}{   LOOP\PYZus{}START\PYZus{}\PYZpc{}d: nop}\PY{l+s+se}{\PYZbs{}n}\PY{l+s}{\PYZdq{}}\PY{p}{,} \PY{n}{endSbl}\PY{p}{)}\PY{p}{;} \PY{c+c1}{// Mark beginning}
        
            \PY{c+c1}{// Procedure}
            \PY{n}{produce\PYZus{}code}\PY{p}{(}\PY{n}{g\PYZus{}node\PYZus{}nth\PYZus{}child}\PY{p}{(}\PY{n}{node}\PY{p}{,} \PY{l+m+mi}{2}\PY{p}{)}\PY{p}{)}\PY{p}{;} \PY{c+c1}{// code}
            \PY{n}{fprintf}\PY{p}{(}\PY{n}{stream}\PY{p}{,} \PY{l+s}{\PYZdq{}}\PY{l+s}{   nop}\PY{l+s+se}{\PYZbs{}n}\PY{l+s}{\PYZdq{}}\PY{p}{)}\PY{p}{;}
        
            \PY{c+c1}{// Increment counter}
            \PY{n}{fprintf}\PY{p}{(}\PY{n}{stream}\PY{p}{,} \PY{l+s}{\PYZdq{}}\PY{l+s}{   LOOP\PYZus{}INCR\PYZus{}\PYZpc{}d: }\PY{l+s}{\PYZdq{}}\PY{p}{,} \PY{n}{endSbl}\PY{p}{)}\PY{p}{;}
            \PY{n}{fprintf}\PY{p}{(}\PY{n}{stream}\PY{p}{,} \PY{l+s}{\PYZdq{}}\PY{l+s}{   nop}\PY{l+s+se}{\PYZbs{}n}\PY{l+s}{\PYZdq{}}\PY{p}{)}\PY{p}{;}
            \PY{n}{fprintf}\PY{p}{(}\PY{n}{stream}\PY{p}{,} \PY{l+s}{\PYZdq{}}\PY{l+s}{   ldloc.s \PYZpc{}d}\PY{l+s+se}{\PYZbs{}n}\PY{l+s}{\PYZdq{}}\PY{p}{,} \PY{n}{endSbl}\PY{p}{)}\PY{p}{;}
            \PY{n}{fprintf}\PY{p}{(}\PY{n}{stream}\PY{p}{,} \PY{l+s}{\PYZdq{}}\PY{l+s}{   ldc.i4.1}\PY{l+s+se}{\PYZbs{}n}\PY{l+s}{\PYZdq{}}\PY{p}{)}\PY{p}{;}
            \PY{n}{fprintf}\PY{p}{(}\PY{n}{stream}\PY{p}{,} \PY{l+s}{\PYZdq{}}\PY{l+s}{   add}\PY{l+s+se}{\PYZbs{}n}\PY{l+s}{\PYZdq{}}\PY{p}{)}\PY{p}{;}
            \PY{n}{fprintf}\PY{p}{(}\PY{n}{stream}\PY{p}{,} \PY{l+s}{\PYZdq{}}\PY{l+s}{   stloc.s \PYZpc{}d}\PY{l+s+se}{\PYZbs{}n}\PY{l+s+se}{\PYZbs{}n}\PY{l+s}{\PYZdq{}}\PY{p}{,} \PY{n}{endSbl}\PY{p}{)}\PY{p}{;} \PY{c+c1}{// Unload counter}
        
            \PY{n}{fprintf}\PY{p}{(}\PY{n}{stream}\PY{p}{,} \PY{l+s}{\PYZdq{}}\PY{l+s}{   LOOP\PYZus{}HEAD\PYZus{}\PYZpc{}d: }\PY{l+s}{\PYZdq{}}\PY{p}{,} \PY{n}{endSbl}\PY{p}{)}\PY{p}{;} \PY{c+c1}{// Mark head}
            \PY{n}{fprintf}\PY{p}{(}\PY{n}{stream}\PY{p}{,} \PY{l+s}{\PYZdq{}}\PY{l+s}{   ldloc.s \PYZpc{}d}\PY{l+s+se}{\PYZbs{}n}\PY{l+s}{\PYZdq{}}\PY{p}{,} \PY{n}{endSbl}\PY{p}{)}\PY{p}{;} \PY{c+c1}{// Reload counter}
            \PY{n}{produce\PYZus{}code}\PY{p}{(}\PY{n}{g\PYZus{}node\PYZus{}nth\PYZus{}child}\PY{p}{(}\PY{n}{node}\PY{p}{,} \PY{l+m+mi}{1}\PY{p}{)}\PY{p}{)}\PY{p}{;} \PY{c+c1}{// expr}
        
            \PY{c+c1}{// Check condition}
            \PY{n}{fprintf}\PY{p}{(}\PY{n}{stream}\PY{p}{,} \PY{l+s}{\PYZdq{}}\PY{l+s}{   cgt}\PY{l+s+se}{\PYZbs{}n}\PY{l+s}{\PYZdq{}}\PY{p}{)}\PY{p}{;}
            \PY{n}{fprintf}\PY{p}{(}\PY{n}{stream}\PY{p}{,} \PY{l+s}{\PYZdq{}}\PY{l+s}{   ldc.i4.0}\PY{l+s+se}{\PYZbs{}n}\PY{l+s}{\PYZdq{}}\PY{p}{)}\PY{p}{;}
            \PY{n}{fprintf}\PY{p}{(}\PY{n}{stream}\PY{p}{,} \PY{l+s}{\PYZdq{}}\PY{l+s}{   ceq}\PY{l+s+se}{\PYZbs{}n}\PY{l+s}{\PYZdq{}}\PY{p}{)}\PY{p}{;}
        
            \PY{n}{fprintf}\PY{p}{(}\PY{n}{stream}\PY{p}{,} \PY{l+s}{\PYZdq{}}\PY{l+s}{   brtrue.s LOOP\PYZus{}START\PYZus{}\PYZpc{}d}\PY{l+s+se}{\PYZbs{}n}\PY{l+s}{\PYZdq{}}\PY{p}{,} \PY{n}{endSbl}\PY{p}{)}\PY{p}{;} \PY{c+c1}{// jump to beginning of loop}
        \PY{k}{if} \PY{n}{cond}
        
            \PY{n}{produce\PYZus{}code}\PY{p}{(}\PY{n}{g\PYZus{}node\PYZus{}nth\PYZus{}child}\PY{p}{(}\PY{n}{node}\PY{p}{,} \PY{l+m+mi}{3}\PY{p}{)}\PY{p}{)}\PY{p}{;} \PY{c+c1}{// endforeach}
            \PY{n}{fprintf}\PY{p}{(}\PY{n}{stream}\PY{p}{,} \PY{l+s}{\PYZdq{}}\PY{l+s}{   // End loop}\PY{l+s+se}{\PYZbs{}n}\PY{l+s}{\PYZdq{}}\PY{p}{)}\PY{p}{;}
            \PY{n}{fprintf}\PY{p}{(}\PY{n}{stream}\PY{p}{,} \PY{l+s}{\PYZdq{}}\PY{l+s}{   LOOP\PYZus{}END\PYZus{}\PYZpc{}d: nop}\PY{l+s+se}{\PYZbs{}n}\PY{l+s}{\PYZdq{}}\PY{p}{,} \PY{n}{endSbl}\PY{p}{)}\PY{p}{;}
        \PY{p}{\PYZcb{}}
\end{Verbatim}

    \hypertarget{mieux-guxe9rer-les-erreurs-de-syntaxe}{%
\section{Mieux gérer les erreurs de
syntaxe}\label{mieux-guxe9rer-les-erreurs-de-syntaxe}}

Si un erreur de syntaxe survient, on n'obtient le message
\texttt{syntax\ error} qui ne rend pas le débogage moins fastidieux.

Inspiré
d'\href{https://www.ibm.com/developerworks/library/l-flexbison/index.html}{une
publication} de Christian HAGEN sur les méthodes de gestion en utilisant
Flex et Bison, on va essayé de produire un système de gestion un peu
plus adéquat.

\hypertarget{bison-plus-bavarde}{%
\subsection{Bison plus bavarde}\label{bison-plus-bavarde}}

On peut commencer en mettant l'option \texttt{YYERROR\_VERBOSE} à 1

\begin{Shaded}
\begin{Highlighting}[]
\PreprocessorTok{#define YYERROR_VERBOSE 1}
\end{Highlighting}
\end{Shaded}

La sortie sera un peu plus détaillée, mais loin d'être suffisant

\begin{Shaded}
\begin{Highlighting}[]
\NormalTok{£: }\FunctionTok{sh}\NormalTok{ test.sh test/errorHandling/ }\CommentTok{# le token DO est absent}

\ExtensionTok{...}
\ExtensionTok{Assembling}\NormalTok{ from facile...         # Extrait de la sortie de test.sh}
\ExtensionTok{syntax}\NormalTok{ error, unexpected TOK_IF   # Pointer vers le mauvais endroit}
\end{Highlighting}
\end{Shaded}

\hypertarget{sous-diviser-la-grammaire-pour-guxe9rer-les-erreurs}{%
\subsection{Sous-diviser la grammaire pour gérer les
erreurs}\label{sous-diviser-la-grammaire-pour-guxe9rer-les-erreurs}}

Par exemple pour la boucle \texttt{foreach} de l'exemple précédent, on
veut vérifier que les tokens/lexèmes(\texttt{foreach}, \texttt{in},
\texttt{..}, \texttt{do}).

Pour ce faire, on va ré-écrire la règle grammaticalle de
\texttt{foreach-stmt} en remplaçant les tokens (symboles terminaux) par
une nouvelle règle de production:

\begin{verbatim}
foreach-stmt:
    chk_foreach ident chk_in expr chk_to expr chk_do code endforeach
    {
        $$ = g_node_new("foreach");
        g_node_append($$, $4);
        g_node_append($$, $6);
        g_node_append($$, $8);
        g_node_append($$, $9);
    }
;
\end{verbatim}

Ensuite, pour chaque nouvelle règle de production, on offre 2
possibilités, soit le token (cas valide), soit l'erreur (cas invalide).
Si on obtient un erreur, on affiche le message avec \texttt{yyerror()}
et quitte le programme avec le macro \texttt{YYABORT}. Le macro est
important car l'instruction \texttt{error} permet le programme de
continuer, et on risque d'avoir un code erroneux qui compile

\begin{verbatim}
chk_in:
    TOK_IN | { yyerror("Missing token <in>"); YYABORT; }   error
;

chk_to:
    TOK_ARR_TO | { yyerror("Missing token <..>"); YYABORT; }   error
;

chk_do:
    TOK_DO | { yyerror("Missing token <do>"); YYABORT; }   error
;

chk_foreach:
    TOK_FOREACH | { yyerror("Missing token <foreach>"); YYABORT; }   error
;
\end{verbatim}

    \hypertarget{conclusion}{%
\section{Conclusion}\label{conclusion}}

Flex et Bison est un outil puissant pour l'analyse syntaxique. La
définition des composants est facile à faire mais la génération du code
CIL nécessite un peu plus de connaissances sur la programmation de bas
niveau. Sa gestion des erreurs de syntaxe est primitive mais peut être
améliorée avec les techniques avancés.


    % Add a bibliography block to the postdoc
    
    
    
    \end{document}
